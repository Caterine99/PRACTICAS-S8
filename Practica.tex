\documentclass{article}\usepackage[]{graphicx}\usepackage[]{xcolor}
% maxwidth is the original width if it is less than linewidth
% otherwise use linewidth (to make sure the graphics do not exceed the margin)
\makeatletter
\def\maxwidth{ %
  \ifdim\Gin@nat@width>\linewidth
    \linewidth
  \else
    \Gin@nat@width
  \fi
}
\makeatother

\definecolor{fgcolor}{rgb}{0.345, 0.345, 0.345}
\newcommand{\hlnum}[1]{\textcolor[rgb]{0.686,0.059,0.569}{#1}}%
\newcommand{\hlstr}[1]{\textcolor[rgb]{0.192,0.494,0.8}{#1}}%
\newcommand{\hlcom}[1]{\textcolor[rgb]{0.678,0.584,0.686}{\textit{#1}}}%
\newcommand{\hlopt}[1]{\textcolor[rgb]{0,0,0}{#1}}%
\newcommand{\hlstd}[1]{\textcolor[rgb]{0.345,0.345,0.345}{#1}}%
\newcommand{\hlkwa}[1]{\textcolor[rgb]{0.161,0.373,0.58}{\textbf{#1}}}%
\newcommand{\hlkwb}[1]{\textcolor[rgb]{0.69,0.353,0.396}{#1}}%
\newcommand{\hlkwc}[1]{\textcolor[rgb]{0.333,0.667,0.333}{#1}}%
\newcommand{\hlkwd}[1]{\textcolor[rgb]{0.737,0.353,0.396}{\textbf{#1}}}%
\let\hlipl\hlkwb

\usepackage{framed}
\makeatletter
\newenvironment{kframe}{%
 \def\at@end@of@kframe{}%
 \ifinner\ifhmode%
  \def\at@end@of@kframe{\end{minipage}}%
  \begin{minipage}{\columnwidth}%
 \fi\fi%
 \def\FrameCommand##1{\hskip\@totalleftmargin \hskip-\fboxsep
 \colorbox{shadecolor}{##1}\hskip-\fboxsep
     % There is no \\@totalrightmargin, so:
     \hskip-\linewidth \hskip-\@totalleftmargin \hskip\columnwidth}%
 \MakeFramed {\advance\hsize-\width
   \@totalleftmargin\z@ \linewidth\hsize
   \@setminipage}}%
 {\par\unskip\endMakeFramed%
 \at@end@of@kframe}
\makeatother

\definecolor{shadecolor}{rgb}{.97, .97, .97}
\definecolor{messagecolor}{rgb}{0, 0, 0}
\definecolor{warningcolor}{rgb}{1, 0, 1}
\definecolor{errorcolor}{rgb}{1, 0, 0}
\newenvironment{knitrout}{}{} % an empty environment to be redefined in TeX

\usepackage{alltt}
\IfFileExists{upquote.sty}{\usepackage{upquote}}{}
\begin{document}

\section{CONTRASTES DE HIPÓTESIS}
\subsection{ Contrastes de hipótesis paramétricos}
\subsubsection{Supuesto Práctico 1}
Con el fin de estudiar el número medio de flexiones continuadas que pueden realizar sus alumnos, un profesor de educación física somete a 75 de ellos, elegidos aleatoriamente, a una prueba. El número de flexiones realizado por cada alumno, así como su sexo y si realizan o no deporte fuera del horario escolar se muestran en el fichero Flexiones.txt.
\begin{knitrout}
\definecolor{shadecolor}{rgb}{0.969, 0.969, 0.969}\color{fgcolor}\begin{kframe}
\begin{alltt}
\hlstd{datos}\hlkwb{<-}\hlkwd{read.table}\hlstd{(}\hlstr{"Flexiones.txt"}\hlstd{,} \hlkwc{header}\hlstd{=}\hlnum{TRUE}\hlstd{)}
\hlstd{datos}
\end{alltt}
\begin{verbatim}
##    Flexiones Sexo Deporte
## 1         60    H       0
## 2         41    H       0
## 3         53    M       1
## 4         53    M       0
## 5         41    H       0
## 6         56    H       0
## 7         50    H       0
## 8         53    M       1
## 9         50    M       1
## 10        48    M       0
## 11        50    M       1
## 12        48    M       1
## 13        56    H       0
## 14        52    M       1
## 15        54    M       0
## 16        50    H       1
## 17        50    H       0
## 18        54    H       0
## 19        52    H       1
## 20        48    H       0
## 21        48    H       1
## 22        35    M       1
## 23        50    M       1
## 24        41    M       1
## 25        56    M       1
## 26        52    M       1
## 27        56    M       0
## 28        54    H       1
## 29        53    H       0
## 30        53    M       0
## 31        53    H       0
## 32        41    M       1
## 33        48    M       0
## 34        50    H       1
## 35        50    M       1
## 36        52    H       0
## 37        53    M       0
## 38        35    H       0
## 39        35    H       0
## 40        54    M       0
## 41        46    M       1
## 42        48    H       0
## 43        50    M       0
## 44        48    H       0
## 45        41    M       0
## 46        48    M       1
## 47        60    H       1
## 48        53    M       0
## 49        54    M       1
## 50        56    H       1
## 51        50    H       1
## 52        41    H       0
## 53        60    M       1
## 54        60    M       1
## 55        54    H       0
## 56        54    H       0
## 57        53    H       0
## 58        35    M       0
## 59        54    H       0
## 60        48    M       0
## 61        50    H       0
## 62        54    H       0
## 63        54    H       0
## 64        53    H       0
## 65        52    H       0
## 66        50    H       0
## 67        52    H       0
## 68        48    H       1
## 69        46    H       1
## 70        53    H       0
## 71        50    H       0
## 72        35    H       0
## 73        50    H       1
## 74        60    M       1
## 75        50    H       0
\end{verbatim}
\end{kframe}
\end{knitrout}

Una vez hecho esto, introducimos en R el nivel de significación que proporciona el enunciado.
\begin{knitrout}
\definecolor{shadecolor}{rgb}{0.969, 0.969, 0.969}\color{fgcolor}\begin{kframe}
\begin{alltt}
\hlstd{alpha}\hlkwb{<-}\hlnum{0.05}
\end{alltt}
\end{kframe}
\end{knitrout}

A continuación, calculamos el valor del estadístico de contraste.
\begin{knitrout}
\definecolor{shadecolor}{rgb}{0.969, 0.969, 0.969}\color{fgcolor}\begin{kframe}
\begin{alltt}
\hlstd{media}\hlkwb{<-}\hlkwd{mean}\hlstd{(datos}\hlopt{$}\hlstd{Flexiones)}
\hlstd{mu_0}\hlkwb{<-}\hlnum{55}
\hlstd{varianza}\hlkwb{<-}\hlnum{7.5}
\hlstd{n}\hlkwb{<-}\hlkwd{nrow}\hlstd{(datos)}
\hlstd{Z}\hlkwb{<-}\hlstd{(media}\hlopt{-}\hlstd{mu_0)}\hlopt{/}\hlstd{(}\hlkwd{sqrt}\hlstd{(varianza)}\hlopt{/}\hlkwd{sqrt}\hlstd{(n))}
\hlstd{Z}
\end{alltt}
\begin{verbatim}
## [1] -15.47408
\end{verbatim}
\end{kframe}
\end{knitrout}

Y también el valor crítico, que en este caso coincide con $Z_{1-\alpha/2}$, el cuantil $1-\alpha/2$ de una distribución normal de media 0 y varianza 1.
\begin{knitrout}
\definecolor{shadecolor}{rgb}{0.969, 0.969, 0.969}\color{fgcolor}\begin{kframe}
\begin{alltt}
\hlstd{cuantil}\hlkwb{<-}\hlkwd{qnorm}\hlstd{(}\hlnum{1}\hlopt{-}\hlstd{alpha}\hlopt{/}\hlnum{2}\hlstd{)}
\hlstd{cuantil}
\end{alltt}
\begin{verbatim}
## [1] 1.959964
\end{verbatim}
\end{kframe}
\end{knitrout}

\end{document}
